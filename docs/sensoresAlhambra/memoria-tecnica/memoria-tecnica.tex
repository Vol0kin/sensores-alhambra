\documentclass[11pt,a4paper]{article}
\usepackage[spanish,es-nodecimaldot]{babel}	% Utilizar español
\usepackage[utf8]{inputenc}					% Caracteres UTF-8
\usepackage{graphicx}						% Imagenes
\usepackage[hidelinks]{hyperref}			% Poner enlaces sin marcarlos en rojo
\usepackage{fancyhdr}						% Modificar encabezados y pies de pagina
\usepackage{float}							% Insertar figuras
\usepackage[textwidth=390pt]{geometry}		% Anchura de la pagina
\usepackage[nottoc]{tocbibind}				% Referencias (no incluir num pagina indice en Indice)
\usepackage{enumitem}						% Permitir enumerate con distintos simbolos
\usepackage[T1]{fontenc}					% Usar textsc en sections
\usepackage{amsmath}						% Símbolos matemáticos
\usepackage[simplified]{pgf-umlcd}
\usepackage{pdflscape}
\usetikzlibrary{babel} % Problemas del español al usar <,> para las citas
\usepackage{typearea} % Paginas horizontales


% Comando para poner el nombre de la asignatura
\newcommand{\asignatura}{Nuevos Paradigmas de Interacción}
\newcommand{\autorv}{Vladislav Nikolov Vasilev}
\newcommand{\autorj}{José María Sánchez Guerrero}
\newcommand{\autorf}{Fernando Vallecillos Ruiz}
\newcommand{\titulo}{Práctica Sensores}
\newcommand{\subtitulo}{Memoria Técnica}


% Configuracion de encabezados y pies de pagina
\pagestyle{fancy}
\lhead{Vladislav, José María, Fernando}
\rhead{\asignatura{}}
\lfoot{Grado en Ingeniería Informática}
\cfoot{}
\rfoot{\thepage}
\renewcommand{\headrulewidth}{0.4pt}		% Linea cabeza de pagina
\renewcommand{\footrulewidth}{0.4pt}		% Linea pie de pagina


% new pagestyle
\fancypagestyle{lscape}{
  \headwidth\textwidth
}

\begin{document}
\pagenumbering{gobble}

% Pagina de titulo
\begin{titlepage}

\begin{minipage}{\textwidth}

\centering

\includegraphics[scale=0.5]{img/ugr.png}\\

\textsc{\Large \asignatura{}\\[0.2cm]}
\textsc{GRADO EN INGENIERÍA INFORMÁTICA}\\[1cm]

\noindent\rule[-1ex]{\textwidth}{1pt}\\[1.5ex]
\textsc{{\Huge \titulo\\[0.5ex]}}
\textsc{{\Large \subtitulo\\}}
\noindent\rule[-1ex]{\textwidth}{2pt}\\[3.5ex]

\end{minipage}

\vspace{0.5cm}

\begin{minipage}{\textwidth}

\centering

\textbf{Autores}\\ {\autorv{}}\\{\autorj{}}\\{\autorf{}}\\[2.5ex]
\textbf{Rama}\\ {Computación y Sistemas Inteligentes}\\[2.5ex]
\vspace{0.3cm}

\includegraphics[scale=0.3]{img/etsiit.jpeg}

\vspace{0.5cm}
\textsc{Escuela Técnica Superior de Ingenierías Informática y de Telecomunicación}\\
\vspace{0.5cm}
\textsc{Curso 2019-2020}
\end{minipage}
\end{titlepage}

\pagenumbering{arabic}
\tableofcontents
\thispagestyle{empty}				% No usar estilo en la pagina de indice

\newpage

\setlength{\parskip}{1em}


\section{Introducción}
En este proyecto vamos a explicar nuestra \textit{Natural User Interface (NUI)} que hará que nuestra visita a la Alhambra sea más
dinámica y productiva. El proyecto constaría de varios dispositivos, como pueden ser principalmente unas gafas de realidad aumentada,
un dispositivo Intel Real Sense y un micrófono integrados en las gafas, y un smartphone que utilizaremos tanto para manejar el sistema
gracias a los múltiples sensores que incorporan, como controlador de la aplicación.

En esta primera versión del proyecto vamos a encargarnos de la interfaz por sensores, es decir, dejaremos a un lado el Intel Real Sense
y el controlador por voz. Vamos a encargarnos de la realidad aumentada y el resto de sensores para manejar el sistema.

Como no disponemos de unas gafas con estas características, simularemos su funcionamiento en la propia aplicación, utilizando un visor
VR con una imagen 3D del patio de los Arrayanes y un lector de código QR. La idea es que estas dos características fuesen una sola, en
la cámara de las gafas. Por otro lado, llevaremos el \textit{smartphone} en la mano para controlar el contenido mostrado en las gafas, y
además, mostrará una maqueta de la Alhambra que nos servirá como mapa.

Por defecto, la aplicación lanza el Main Activity, inicializando la aplicación y mostrando el ViewARFragment (nuestro visor VR). El
visor muestra una imagen de 360º, en la cual, cuando estemos mirando una estructura en concreto (ya sea un edificio, puertas, columnas,
jarrones, etc.), se nos marcará indicando que podremos interaccionar con ella. Si movemos ligeramente el móvil hacia abajo cuando una de
ellas esté resaltada, iniciamos la InfoActivity, que es una pequeña página que muestra información extra sobre la estructura, como si fuese
una enciclopedia. En nuestra aplicación esto puede suponer un problema, ya que el visor lo tenemos en el móvil y al realizar el movimiento
podemos deseleccionar lo que teníamos resaltado; sin embargo, como esto estaría integrado en las gafas, cuando movamos el móvil, las gafas
(que es donde vemos el objeto) no se verán afectadas.

Una vez en el InfoActivity, para volver al visor VR, también realizaremos un movimiento horizontal con nuestro dispositivo. En nuestra
aplicación, esto sale a partir del ViewARFragment, pero en sistema definitivo saldría a través de las gafas.

En el menú desplegable, también tenemos la opción de Camera, que nos lleva al CameraFragment. Este implementa el lector de códigos QR, el
cual nos servirá a la hora de entrar a un edificio. Habrá códigos QR en cada uno y se obtendrá un plano del edificio ya que las señales
GPS pueden fallar o simplemente que el edificio tenga varias plantas y necesitemos ubicarnos mejor. Al igual que antes, para volver a la
cámara, realizaremos un movimiento horizontal con nuestro dispositivo.


% start new page before setting page layout,
% otherwise previous page is also affected
\KOMAoption{paper}{landscape}%
\typearea{12}% sets new DIV

% Establecer pagina horizontal
\recalctypearea
% needed to show page in landscape in viewer
\pdfpageheight=\paperheight
\pdfpagewidth=\paperwidth
% Poner estilo
\pagestyle{lscape}

\newpage

\section{Diagrama de clases}

A continuación se puede ver el diagrama de clases. No se han incluido atributos ni métodos porque se van a describir más adelante.

\begin{figure}[H]
\centering
\begin{tikzpicture}[scale=0.7]
	\begin{class}[text width=3cm]{MainActivity}{0,0}
	\end{class}
	
	\begin{class}[text width=4cm]{CameraFragment}{-9, -5}
	\end{class}
	\begin{class}[text width=3cm]{ExitFragment}{-7, -1}
	\end{class}
	\begin{class}[text width=5cm]{NavigationFragment}{0, -6.5}
	\end{class}
	\begin{class}[text width=4cm]{ViewARFragment}{9, -5}
	\end{class}
	\begin{class}[text width=4cm]{BlueprintsActivity}{-9, -9}
	\end{class}
	\begin{class}[text width=3cm]{InfoActivity}{9, -9}
	\end{class}
	
	\unidirectionalAssociation{MainActivity}{}{1}{CameraFragment}
	\unidirectionalAssociation{MainActivity}{}{1}{ExitFragment}
	\unidirectionalAssociation{MainActivity}{}{1}{NavigationFragment}
	\unidirectionalAssociation{MainActivity}{}{1}{ViewARFragment}
	
	\draw[umlcd style dashed line,->] (ViewARFragment) -- node[above, sloped, black] {$<<$uses$>>$} (InfoActivity);
	\draw[umlcd style dashed line,->] (CameraFragment) -- node[above, sloped, black] {$<<$uses$>>$} (BlueprintsActivity);
\end{tikzpicture}
\caption{Diagrama de clases simplificado.}
\label{fig:class-diagram}
\end{figure}

\KOMAoptions{paper=portrait}
\recalctypearea
\pdfpageheight=\paperheight
\pdfpagewidth=\paperwidth
\headwidth\textwidth

\section{Descripción de las clases}

Una vez visto el diagrama de clases, vamos a proceder a comentar brevemente qué es lo que hace cada clase:

\begin{itemize}
	\item \textbf{MainActivity}: Es la actividad principal, encargada de construir la barra de actividades que permite acceder
	a los \textit{fragments} y de solicitar los permisos de cámara y localización necesarios para ejecutar la aplicación.
	\item \textbf{ExitFragment}: \textit{Fragment} que permite salir de la aplicación.
	\item \textbf{CameraFragment}: \textit{Fragment} que permite acceder al sensor de la cámara para poder realizar la lectura
	de códigos QR. Se encarga de leerlos y pasarlos a \textbf{BlueprintActivity} para que éste los procese. Para la detección y
	lectura de los códigos QR se utiliza la biblioteca \textit{BarcodeDetector}.
	\item \textbf{NavigationFragment}: \textit{Fragment} que permite acceder a un mapa de la Alhambra y navegar por él, destacando
	algunos de los edificios y permitiendo interactuar con ellos. Se utiliza la API de GoogleMaps para trabajar con mapas.
	\item \textbf{ViewARFragment}: \textit{Fragment} que simula la visión en realidad aumentada del interior de los Palacios
	Nazaríes. Simula que detecta zonas interesantes y las destaca. Cuando se realiza un gesto de \textit{shake} con el móvil,
	muestra más información sobre el elemento destacado, utilizando para ello la clase \textbf{InfoActivity}.
	\item \textbf{BlueprintsActivity}: Actividad que se encarga de recibir las lecturas QR de \textbf{CameraFragment} y de
	procesarlas, mostrando la información correspondiente a la lectura.
	\item \textbf{InfoActivity}: Clase que muestra información sobre la zona que ha sido destacada en \textbf{ViewARFragment}.
	Procesa la información que tiene ésta y muestra una u otra información.
\end{itemize}

\section{Atributos de las clases}

\section{Métodos de las clases}

\newpage

\begin{thebibliography}{5}

\bibitem{nombre-referencia}
Texto referencia
\\\url{https://url.referencia.com}

\end{thebibliography}

\end{document}

